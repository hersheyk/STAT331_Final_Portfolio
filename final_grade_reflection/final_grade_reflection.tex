% Options for packages loaded elsewhere
\PassOptionsToPackage{unicode}{hyperref}
\PassOptionsToPackage{hyphens}{url}
\PassOptionsToPackage{dvipsnames,svgnames,x11names}{xcolor}
%
\documentclass[
  letterpaper,
  DIV=11,
  numbers=noendperiod]{scrartcl}

\usepackage{amsmath,amssymb}
\usepackage{lmodern}
\usepackage{iftex}
\ifPDFTeX
  \usepackage[T1]{fontenc}
  \usepackage[utf8]{inputenc}
  \usepackage{textcomp} % provide euro and other symbols
\else % if luatex or xetex
  \usepackage{unicode-math}
  \defaultfontfeatures{Scale=MatchLowercase}
  \defaultfontfeatures[\rmfamily]{Ligatures=TeX,Scale=1}
  \setmainfont[]{Times New Roman}
\fi
% Use upquote if available, for straight quotes in verbatim environments
\IfFileExists{upquote.sty}{\usepackage{upquote}}{}
\IfFileExists{microtype.sty}{% use microtype if available
  \usepackage[]{microtype}
  \UseMicrotypeSet[protrusion]{basicmath} % disable protrusion for tt fonts
}{}
\makeatletter
\@ifundefined{KOMAClassName}{% if non-KOMA class
  \IfFileExists{parskip.sty}{%
    \usepackage{parskip}
  }{% else
    \setlength{\parindent}{0pt}
    \setlength{\parskip}{6pt plus 2pt minus 1pt}}
}{% if KOMA class
  \KOMAoptions{parskip=half}}
\makeatother
\usepackage{xcolor}
\setlength{\emergencystretch}{3em} % prevent overfull lines
\setcounter{secnumdepth}{-\maxdimen} % remove section numbering
% Make \paragraph and \subparagraph free-standing
\ifx\paragraph\undefined\else
  \let\oldparagraph\paragraph
  \renewcommand{\paragraph}[1]{\oldparagraph{#1}\mbox{}}
\fi
\ifx\subparagraph\undefined\else
  \let\oldsubparagraph\subparagraph
  \renewcommand{\subparagraph}[1]{\oldsubparagraph{#1}\mbox{}}
\fi


\providecommand{\tightlist}{%
  \setlength{\itemsep}{0pt}\setlength{\parskip}{0pt}}\usepackage{longtable,booktabs,array}
\usepackage{calc} % for calculating minipage widths
% Correct order of tables after \paragraph or \subparagraph
\usepackage{etoolbox}
\makeatletter
\patchcmd\longtable{\par}{\if@noskipsec\mbox{}\fi\par}{}{}
\makeatother
% Allow footnotes in longtable head/foot
\IfFileExists{footnotehyper.sty}{\usepackage{footnotehyper}}{\usepackage{footnote}}
\makesavenoteenv{longtable}
\usepackage{graphicx}
\makeatletter
\def\maxwidth{\ifdim\Gin@nat@width>\linewidth\linewidth\else\Gin@nat@width\fi}
\def\maxheight{\ifdim\Gin@nat@height>\textheight\textheight\else\Gin@nat@height\fi}
\makeatother
% Scale images if necessary, so that they will not overflow the page
% margins by default, and it is still possible to overwrite the defaults
% using explicit options in \includegraphics[width, height, ...]{}
\setkeys{Gin}{width=\maxwidth,height=\maxheight,keepaspectratio}
% Set default figure placement to htbp
\makeatletter
\def\fps@figure{htbp}
\makeatother

\KOMAoption{captions}{tableheading}
\makeatletter
\makeatother
\makeatletter
\makeatother
\makeatletter
\@ifpackageloaded{caption}{}{\usepackage{caption}}
\AtBeginDocument{%
\ifdefined\contentsname
  \renewcommand*\contentsname{Table of contents}
\else
  \newcommand\contentsname{Table of contents}
\fi
\ifdefined\listfigurename
  \renewcommand*\listfigurename{List of Figures}
\else
  \newcommand\listfigurename{List of Figures}
\fi
\ifdefined\listtablename
  \renewcommand*\listtablename{List of Tables}
\else
  \newcommand\listtablename{List of Tables}
\fi
\ifdefined\figurename
  \renewcommand*\figurename{Figure}
\else
  \newcommand\figurename{Figure}
\fi
\ifdefined\tablename
  \renewcommand*\tablename{Table}
\else
  \newcommand\tablename{Table}
\fi
}
\@ifpackageloaded{float}{}{\usepackage{float}}
\floatstyle{ruled}
\@ifundefined{c@chapter}{\newfloat{codelisting}{h}{lop}}{\newfloat{codelisting}{h}{lop}[chapter]}
\floatname{codelisting}{Listing}
\newcommand*\listoflistings{\listof{codelisting}{List of Listings}}
\makeatother
\makeatletter
\@ifpackageloaded{caption}{}{\usepackage{caption}}
\@ifpackageloaded{subcaption}{}{\usepackage{subcaption}}
\makeatother
\makeatletter
\@ifpackageloaded{tcolorbox}{}{\usepackage[many]{tcolorbox}}
\makeatother
\makeatletter
\@ifundefined{shadecolor}{\definecolor{shadecolor}{rgb}{.97, .97, .97}}
\makeatother
\makeatletter
\makeatother
\ifLuaTeX
  \usepackage{selnolig}  % disable illegal ligatures
\fi
\IfFileExists{bookmark.sty}{\usepackage{bookmark}}{\usepackage{hyperref}}
\IfFileExists{xurl.sty}{\usepackage{xurl}}{} % add URL line breaks if available
\urlstyle{same} % disable monospaced font for URLs
\hypersetup{
  pdftitle={Final Grade Reflection},
  pdfauthor={Harshini Karthikeyan},
  colorlinks=true,
  linkcolor={blue},
  filecolor={Maroon},
  citecolor={Blue},
  urlcolor={Blue},
  pdfcreator={LaTeX via pandoc}}

\title{Final Grade Reflection}
\author{Harshini Karthikeyan}
\date{December 3, 2022}

\begin{document}
\maketitle
\ifdefined\Shaded\renewenvironment{Shaded}{\begin{tcolorbox}[frame hidden, interior hidden, sharp corners, boxrule=0pt, borderline west={3pt}{0pt}{shadecolor}, breakable, enhanced]}{\end{tcolorbox}}\fi

I believe I deserve an A for this class because I have completed all the
assignments and received full credit while also consistently going
beyond assigned requirements. I have demonstrated an understanding of a
vast majority of learning targets in Labs and classwork.~

\hypertarget{working-with-data}{%
\subsubsection{Working with Data}\label{working-with-data}}

\textbf{WD-1:} Lab 4 second intro code chunk(labeled `here')~

\textbf{WD-2:} Lab 3 Question 6(code chunk 6: demographics)

\textbf{WD-3:} Practice Activity College Part 2

\textbf{WD-4:} Practice Activity College Part 1~

\textbf{WD-5}: Preview Activity Cereal code chunks; right join and full
join.~

\textbf{WD-6}: Lab 4 code chunk/section; semi joins

\textbf{WD-7}:~

\textbf{Pivot wider:} Lab 9 code chunk 2: ``summarize''

\textbf{Pivot longer:} Preview Activity Cereal, code chunk 1

\hypertarget{reproducibility}{%
\subsubsection{Reproducibility}\label{reproducibility}}

\textbf{R-1}: Lab 9, yaml, code chunk: set up. It is in an r project
labeled ``STAT331\_Final\_Portfolio''.

\textbf{R-2:} Lab 2 numbers 7 \& 8; it is actually shown throughout as
it is formatting, but two examples are those questions and the sections
surrounding them.~

\textbf{R-3}: Lab 3, Section: Familiar Words, usage of slice\_min() and
slice\_max()

\hypertarget{data-visualization-summarization}{%
\subsubsection{Data Visualization \&
Summarization}\label{data-visualization-summarization}}

\textbf{DVS}-\textbf{1}: Lab 9; Section: Summarizing \& Visualizing the
Number of Allisons and Section: Modeling the Numbers

\textbf{DVS-2}: Lab 9; Section Summarizing \& Visualizing the Number of
Allisons

\textbf{DVS}-\textbf{3:} Lab 9; Section Summarizing \& Visualizing the
Number of Allisons

\textbf{DVS-4}: I can calculate numerical summaries of variables.

\textbf{DVS-5:} Lab 3 Question 6(code chunk 6: demographics)

\textbf{DVS-6:} Challenge 9, Section: Allison Table

\textbf{DVS}-\textbf{7}: Challenge 9, Section: Allison Table

\hypertarget{program-efficiency}{%
\subsubsection{Program Efficiency}\label{program-efficiency}}

\textbf{PE-1}:Practice Activity College; Part One; Code chunk; combine

\textbf{PE-2}: Lab 7, Part Two, Task 3 and 5

\textbf{PE-3}: Lab 7, Part One, Task 1 and 2; when I use the across()
function.

\textbf{PE-4}: Practice Activity College; Part One; Code chunk; combine

\hypertarget{data-simulation-modeling}{%
\subsubsection{Data Simulation \&
Modeling}\label{data-simulation-modeling}}

\textbf{DSM-1:} Lab 9; Section: Modeling the Numbers, Code Chunk after
Question 6

\textbf{DSM-2:} Lab 9; Section: Modeling the Numbers, Code chunk BEFORE
Question 6 and Question 6

I have completed every single assignment with full honest attempts and
have submitted revisions whenever they were necessary. which I believe
demonstrates my commitment to continued learning. I have provided two
examples of my revision work and revision reflections. I have done more
than one Challenge option when given a chance for Challenge \#2. Most of
my rendered HTML is colorful due to my chosen YAML colors. For Lab 5, I
did more than what was required, adding a graph that removes the legend,
as William Chase suggested, and instead overlays the text over the graph
itself. I also added a table of contents to the lab. I learned the
pivot\_longer() function before we discussed it in class. I have
repeatedly gone above and beyond regarding my graph formatting(color,
font, positioning, annotating, axis labels, etc.), as demonstrated in
Lab 9, as well as my HTML layout.~

One goal I've fulfilled, which might seem a given for most people, but
can be difficult for me sometimes, is that I have gone to all but one
Stat 331 class. I had a lot of difficulty doing this with classes last
quarter and struggled with motivation, but I have gone to nearly every
class, so I am proud of myself.~ As a team member, I've grown more
comfortable with my classmates and feel no hesitation in texting them if
I have a question or answering others' questions. We call to work
together outside of class, typically to work on labs. I have checked
others' code for bugs, and they have checked mine. Often it is a very
small syntax error that caused the issues. Other times we have had to
turn to Google or the R database to learn more about a particular
function. I have also provided thorough Peer Code reviews. As FaceTime
calls are more difficult to show, I mostly included texts between
different classmates and me, along with a Peer Code Review.

Regarding project timelines, I often submit them after canvas due dates
but before grading deadlines. They usually are only a couple of hours
late. If they are beyond more strict deadlines, I do my best to
communicate as such in advance. I appreciate the soft deadlines as I can
better organize my time concerning my other classwork, which has
stricter deadlines. I make sure to review and read the preview
coursework before class. While sometimes it is after 8 am, it is often
submitted before 12. However, I believe I submitted two of the Preview
Activities relatively late.~

Overall, I believe the work I have done for this class best fits the
category of an A.~



\end{document}
